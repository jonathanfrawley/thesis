\title{Procedurally Generated Indoor Environments for WebGL : Literature Report}
\author{
	Jonathan Frawley \\
	Msc IET \\
	School of Computer Science and Statistics \\
	Trinity College Dublin\\
}
\date{\today}

\documentclass[12pt]{article}

\usepackage{url}

\begin{document}
\maketitle

\clearpage

\begin{abstract}
This is the paper's abstract \ldots
\end{abstract}

\clearpage

\section{Introduction}

\section{Procedural Techniques in Games}

\subsection{Hack 'n' Slash}
Dungeons 'n' stuff

\section{Procedural Content Generation}

\subsection{City Generation}
Recently, there has been much interest in the procedural generation of cities.
Ma\"{i}m et al.~\cite{maim2007populating} demonstrated how it is possible to recreate the population of historic cities using procedural techniques.
The use of procedural techniques to generate the crowds in Pompeii allowed realtime simulation with great variety in character representation.
This would not have been possible with traditional techniques.

The Metropolis project~\cite{web:metropolis} investigates the simulation of crowds in a modern city context.
Members of the crowd are modelled based on a variation of some template.
Different clothes are applied to the same templates to give the illusion of variation in the character models~\cite{mcdonnell2007pipeline}.
People are represented as agents which react to their environment~\cite{ulicny2002towards}.
In this way crowds are simulated using a variety of procedural techniques.

CityEngine~\cite{parish2001procedural} is an example of how procedural techniques can be applied to generates environments of great depth.
Using a combination of extended L-systems and self-sensitive L-systems, roads are generated which realistic simulate that of a target city.
Plans of buildings are generated between road segments in a recursive subdivision scheme, which discards inaccessible buildings.
The models of buildings are generated using an L-system using the bounding box of the building as the axiom of the L-system.
A technique known as \emph{layered grids} is proposed for the procedural generation of interesting textures for buildings.

\subsection{Plans of buildings}
\cite{so}

\section{Demoscene}

\section{OpenGL}

\subsection{OpenGL ES}

\subsection{WebGL}

\subsubsection{Javascript}

\subsubsection{Coffeescript}

\subsubsection{Processing.js}

\subsubsection{Gwt}

\bibliographystyle{abbrv}
\bibliography{main}

\end{document}
