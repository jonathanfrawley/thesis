WebGL enables a new generation of web applications which have access to hardware-accelerated 3D graphics.
This has impact in the area of real-time applications such as computer
games, which are seeing an increasingly large focus on web applications.

A key challenge in making 3D games for web browsers is the creation
and transmission of assets.
Large assets such as meshes and textures, characteristic
of most modern AAA games, present a challenge for web browsers
which typically work with small data sets.

Procedural content generation involves the generation of
assets using mathematical formulae.
The benefit of doing this is that assets do not have to be
physically stored or loaded, but are generated by the application
itself.
Games such as .kkrieger do this successfully to create scenes similar in quality to modern titles such as Doom 3, with very small file size.

This dissertation explores the use of procedural techniques for asset generation in WebGL for indoor environments.
3D geometric information of the areas of a building are generated using 2D procedural plan generation with extrusion into 3D.
Textures are generated using procedural techniques such as bump mapping and Perlin noise.
Two sample scenes are developed of similar complexity :
One which uses procedural content generation and one which loads in assets.
The scenes are compared against other with respect to performance, aesthetic quality and ease of creation.
