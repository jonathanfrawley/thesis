%%%%%%%%%%%%%%%%%%%%%%%%%%%%%%%%%%%%%%%%%%%%%%%%%%%%%%%%%%%%%%%%%%%%%%%%%%%%%
%%%
%%% File: utthesis2.doc, version 2.0jab, February 2002
%%%
%%% Based on: utthesis.doc, version 2.0, January 1995
%%% =============================================
%%% Copyright (c) 1995 by Dinesh Das.  All rights reserved.
%%% This file is free and can be modified or distributed as long as
%%% you meet the following conditions:
%%%
%%% (1) This copyright notice is kept intact on all modified copies.
%%% (2) If you modify this file, you MUST NOT use the original file name.
%%%
%%% This file contains a template that can be used with the package
%%% utthesis.sty and LaTeX2e to produce a thesis that meets the requirements
%%% of the Graduate School of The University of Texas at Austin.
%%%
%%% All of the commands defined by utthesis.sty have default values (see
%%% the file utthesis.sty for these values).  Thus, theoretically, you
%%% don't need to define values for any of them; you can run this file
%%% through LaTeX2e and produce an acceptable thesis, without any text.
%%% However, you probably want to set at least some of the macros (like
%%% \thesisauthor).  In that case, replace "..." with appropriate values,
%%% and uncomment the line (by removing the leading %'s).
%%%
%%%%%%%%%%%%%%%%%%%%%%%%%%%%%%%%%%%%%%%%%%%%%%%%%%%%%%%%%%%%%%%%%%%%%%%%%%%%%

\documentclass[a4paper, 12pt, oneside]{report}         %% LaTeX2e document.
\usepackage {tcdthesis}              %% Preamble.

\usepackage{hyperref} %This is for the cool links to parts of dissertation
\usepackage{epsfig}
\usepackage{amsmath}

\usepackage{url}
\usepackage{graphicx}
%\usepackage{subfig}
\usepackage{caption} %replaces subfig
\usepackage{subcaption} %replaces subfig



%--------------------------------------------------------------
\usepackage{color}
\usepackage{xcolor}
 \usepackage{listings}
  \usepackage{courier}
 \lstset{
         basicstyle=\footnotesize\ttfamily, % Standardschrift
         %numbers=left,               % Ort der Zeilennummern
         numberstyle=\tiny,          % Stil der Zeilennummern
         %stepnumber=2,               % Abstand zwischen den Zeilennummern
         numbersep=5pt,              % Abstand der Nummern zum Text
         tabsize=2,                  % Groesse von Tabs
         extendedchars=true,         %
         breaklines=true,            % Zeilen werden Umgebrochen
         keywordstyle=\color{red},
    		frame=b,         
 %        keywordstyle=[1]\textbf,    % Stil der Keywords
 %        keywordstyle=[2]\textbf,    %
 %        keywordstyle=[3]\textbf,    %
 %        keywordstyle=[4]\textbf,   \sqrt{\sqrt{}} %
         stringstyle=\color{white}\ttfamily, % Farbe der String
         showspaces=false,           % Leerzeichen anzeigen ?
         showtabs=false,             % Tabs anzeigen ?
         xleftmargin=17pt,
         framexleftmargin=17pt,
         framexrightmargin=5pt,
         framexbottommargin=4pt,
         %backgroundcolor=\color{lightgray},
         showstringspaces=false      % Leerzeichen in Strings anzeigen ?        
 }
 \lstloadlanguages{% Check Dokumentation for further languages ...
         %[Visual]Basic
         %Pascal
         %C
         %C++
         %XML
         %HTML
         Java
 }
    %\DeclareCaptionFont{blue}{\color{blue}} 

  %\captionsetup[lstlisting]{singlelinecheck=false, labelfont={blue}, textfont={blue}}
\usepackage{caption}
\DeclareCaptionFont{white}{\color{white}}
\DeclareCaptionFormat{listing}{\colorbox[cmyk]{0.43, 0.35, 0.35,0.01}{\parbox{\textwidth}{\hspace{15pt}#1#2#3}}}
\captionsetup[lstlisting]{format=listing,labelfont=white,textfont=white, singlelinecheck=false, margin=0pt, font={bf,footnotesize}}
%--------------------------------------------------------------


\mastersthesis                     %% Uncomment one of these; if you don't
% \phdthesis                         %% use either, the default is \phdthesis.

%\thesisdraft                       %% Uncomment this if you want a draft
                                     %% version; this will print a timestamp
                                     %% on each page of your thesis.

\leftchapter                       %% Uncomment one of these if you want
%\centerchapter                      %% left-justified, centered or
% \rightchapter                      %% right-justified chapter headings.
                                     %% Chapter headings includes the
                                     %% Contents, Acknowledgments, Lists
                                     %% of Tables and Figures and the Vita.
                                     %% The default is \centerchapter.

% \singlespace                       %% Uncomment one of these if you want
\oneandhalfspace                   %% single-spacing, space-and-a-half
% \doublespace                       %% or double-spacing; the default is
                                     %% \oneandhalfspace, which is the
                                     %% minimum spacing accepted by the
                                     %% Graduate School.

\renewcommand{\thesisauthor}{Jonathan Frawley}            %% Your official UT name.
\renewcommand{\thesismonth}{August}                  %% Your month of graduation.
\renewcommand{\thesisyear}{2011}                      %% Your year of graduation.
\renewcommand{\thesistitle}{Procedural Content Generation of Indoor Environments for WebGL}            %% The title of your thesis; use mixed-case.
\renewcommand{\thesisauthorpreviousdegrees}{B.A(mod) in Computer Science}  %% Your previous degrees, abbreviated; separate multiple degrees by commas.
\renewcommand{\thesissupervisor}{Dr. John Dingliana}      %% Your thesis supervisor; use mixed-case and don't use any titles or degrees.
% \renewcommand{\thesiscosupervisor}{}                %% Your PhD. thesis co-supervisor; if any.

% \renewcommand{\thesiscommitteemembera}{}
% \renewcommand{\thesiscommitteememberb}{}
% \renewcommand{\thesiscommitteememberc}{}
% \renewcommand{\thesiscommitteememberd}{}
% \renewcommand{\thesiscommitteemembere}{}
% \renewcommand{\thesiscommitteememberf}{}
% \renewcommand{\thesiscommitteememberg}{}
% \renewcommand{\thesiscommitteememberh}{}
% \renewcommand{\thesiscommitteememberi}{}


\renewcommand{\thesisauthoraddress}{Trinity College Dublin}

\renewcommand{\thesisdedication}{This is dedicated to my parents, who have supported me throughout my college years.}     %% Your dedication, if you have one; use "\\" for linebreaks.


%%%%%%%%%%%%%%%%%%%%%%%%%%%%%%%%%%%%%%%%%%%%%%%%%%%%%%%%%%%%%%%%%%%%%%%%%%%%%
%%%
%%% The following commands are all optional, but useful if your requirements
%%% are different from the default values in utthesis.sty.  To use them,
%%% simply uncomment (remove the leading %) the line(s).

% \renewcommand{\thesiscommitteesize}{...}
                                     %% Uncomment this only if your thesis
                                     %% committee does NOT have 5 members
                                     %% for \phdthesis or 2 for \mastersthesis.
                                     %% Replace the "..." with the correct
                                     %% number of members.

\renewcommand{\thesisdegree}{Master of Science in Computer Science (Interactive Entertainment Technology)}  
                                     %% default is "DOCTOR OF PHILOSOPHY"
                                     %% for \phdthesis or "MASTER OF ARTS"
                                     %% for \mastersthesis.  Provide the
                                     %% correct FULL OFFICIAL name of
                                     %% the degree.

\renewcommand{\thesisdegreeabbreviation}{M.Sc.}
                                     %% Use this if you also use the above
                                     %% command; provide the OFFICIAL
                                     %% abbreviation of your thesis degree.
%\renewcommand{\thesistype}{Dissertation}    %% Use this ONLY if your thesis type
\renewcommand{\thesistype}{\mastersthesis}    %% Use this ONLY if your thesis type
                                     %% is NOT "Thesis" for \phdthesis
                                     %% or \mastersthesis.
                                     %% Provide the OFFICIAL type of the
                                     %% thesis; use mixed-case.

% \renewcommand{\thesistypist}{...}  %% Use this to specify the name of
                                     %% the thesis typist if it is anything
                                     %% other than "the author".

%%%
%%%%%%%%%%%%%%%%%%%%%%%%%%%%%%%%%%%%%%%%%%%%%%%%%%%%%%%%%%%%%%%%%%%%%%%%%%%%%

\begin{document}                                  %% BEGIN THE DOCUMENT

\thesistitlepage                                  %% Generate the title page.

\thesisdeclarationpage                %% Generate the declaration page.

\thesispermissionpage                 %% Generate the copyright permission page

%\thesisdedicationpage                             %% Generate the dedication page.

\begin{thesisacknowledgments}                     %% Use this to write your
I would like to thank firstly my supervisor, Dr. John Dingliana, for not only his constant support and advice during my project, but also for his role as course director, which he performed admirably.
I would also like to thank my parents who have been amazing in supporting me throughout my college years.
I would also like to thank everyone at DemonWare, who have provided me with feedback on my project.
Finally, I would like to thank Aoife, my girlfriend, for helping me work to finish this thesis, and for supporting me in everything I do.
\end{thesisacknowledgments}                       %% allowed in LaTeX2e par-mode.

\begin{thesisabstract}
WebGL enables a new generation of web applications which have access to hardware-accelerated 3D graphics.
This has impact in the area of real-time applications such as computer
games, which are seeing an increasingly large focus on web applications.

A key challenge in making 3D games for web browsers is the creation
and transmission of assets.
Large assets such as meshes and textures, characteristic
of most modern AAA games, present a challenge for web browsers
which typically work with small data sets.

Procedural content generation involves the generation of
assets using mathematical formulae.
The benefit of doing this is that assets do not have to be
physically stored or loaded, but are generated by the application
itself.
Games such as .kkrieger do this successfully to create scenes similar in quality to modern titles such as Doom 3, with very small file size.

This dissertation explores the use of procedural techniques for asset generation in WebGL for indoor environments.
3D geometric information of the areas of a building are generated using 2D procedural plan generation with extrusion into 3D.
Textures are generated using procedural techniques such as bump mapping and Perlin noise.
Two sample scenes are developed of similar complexity :
One which uses procedural content generation and one which loads in assets.
The scenes are compared against other with respect to performance, aesthetic quality and ease of creation.

\end{thesisabstract}

\tableofcontents                                  %% Generate table of contents.
\listoftables                                     %% Uncomment this to generate list of tables.
\listoffigures                                    %% Uncomment this to generate list of figures.

%%
%% Include thesis chapters here...
%%
  \chapter{Introduction}
Currently in the game industry, assets are generally created by artists, not programmers.
These assets are usually in the form of files, generally quite large in size.
In 3D games, assets generally refer to 3D models and textures.
Generally, the way games are structured is that assets are loaded into the game using model and texture loaders, and the game logic operates on these models and textures to present a realistic environment.

A key problem with this approach is the file sizes that result from model creation and texture creation, especially in modern games where the models and textures could take up hundreds of megabytes of storage. 
This is seen in the ``mod packs'' provided by fans of games to replace the models and textures in games, where a single model can be over 4 megabytes in size~\cite{web:oblivionmodpack}.
With demand in the game industry for increasing complexity in graphics, as well as an increased usage of low power platforms such as mobile devices and the web, a problem exists where players will either have low-resolution models or long-load times for games.

Procedural content generation allows us to generate assets in a procedural way, using algorithms and mathematical formulae.
In this way, models and textures are represented using a very small amount of code, which can be seen as a form of compression.
The Demoscene has proven that procedural generation of diverse content is possible with a very small program size~\cite{web:demoscene4k}.
With procedural content generation, we are able to represent the same quality of models with a much smaller size of file, which is crucial for the web, where all content needs to be transmitted over the network.

This dissertation presents an approach for incorporating procedural content generated content into a traditional game's pipeline.
We will show how it is possible to provide assets in a generic fashion, so that either procedural or static assets can be used.
We also present a vertical slice prototype of this general design which genereates content procedurally, which we compare against another prototype which uses static assets.
By comparing the two prototypes in identical scenes, we can make an informed approach as to why procedural content generation is superior in some circumstances to loading in static assets.



I undertook this project due to my interest in assessing WebGL as a platform for games development.
Upon hearing about WebGL, I was very interested to research the challenges that would be involved to bring interactive 3D games to this new platform.

I also have a fascination with procedural techniques, in particular the Demoscene~\cite{web:demoscene4k}.
The generation of scenes of high complexity with a very small executable size is something which I find very interesting.
Outside of admiring the techniques used by the Demoscene coders, I had never researched procedural techniques before.

During the IET course, I gained a lot of experience with desktop OpenGL, and had some experience with GLSL.
I had no experience with WebGL before staring this project, and it took a while to understand the limitations there were compared to desktop OpenGL.

My research question of how to implement procedural content generation in WebGL came as the result of my initial research into the topic, where it was very difficult to find details on how to incorporate procedural techniques into a codebase, or how to use it in conjunction with static assets.

I will now outline the chapters which are to follow:
\begin{itemize}
	\item Chapter~\ref{ch:stateoftheart}: State-of-the-art - This chapter will describe the state of the art in terms of WebGL and procedural content generation.
	\item Chapter~\ref{ch:design}: Design - The design of the general framework is presented and also the design of the prototype applications developed.
	\item Chapter~\ref{ch:impl}: Implementation - Describes the implementation of the prototype and the design application.
	\item Chapter~\ref{ch:evaluation}: Evaluation - Presents the results of performance analysis and quantative analysis of the prototype created and the design developed.
	\item Chapter~\ref{ch:conclusions}: Conclusions And Future Work - Account of the results of the work done in the project and the research contributions made, future work which could expand on the project.
\end{itemize}

  \chapter{State Of The Art}
\label{ch:stateoftheart}
Today the realm of computer graphics has changed dramatically to what it was ten years ago.
Whereas 10 years ago the only devices which supported 3D graphics were either desktop computers or custom-designed computer games consoles.

On June 2007, Apple release its iPhone to the world, which included a PowerVR MBX chip~\cite{web:powervrmbx}.
An API was developed to enable mobile developers to use OpenGL on mobile applications, where resources are limited.
The OpenGL ES 1.1 API was developed which enabled the fixed function OpenGL pipeline on mobile devices~\cite{web:opengles11}.
This then evolved into the OpenGL ES 2.0 API, which allowed for programmable shaders like desktop OpenGL~\cite{web:opengles20}.
OpenGL ES 2.0 is supported by the new PowerVR SGX chip and is in current state of the art devices such as the Nokia N900 and the Apple iPhone 4~\cite{web:powervrsgx}.
This increase in power is due to an increased demand on users for games and other graphical applications on mobile devices.

As well as mobile devices gaining increased power and popularity, a similar movement is taking place at the same time in the web domain.
HTML, the markup language used for the web, is undergoing a significant review with HTML5~\cite{web:html5}.
HTML5 is an open standard for web applications which aims to replace existing plugins such as Adobe's flash which offer similar functionality.
One of the areas where flash is immensely popular is the world of online games.
This is shown by websites such as Kongregate, which have around 42,000 people playing their games at the time of writing~\cite{web:kongregate}.
Sites such as html5games~\cite{web:html5games} aims to capture this market using HTML5 as the technology.

HTML5 does not define a standard for 3D graphics, although a standard known as WebGL has been developed to fill this gap.

\section{WebGL}
WebGL is a 3D rendering API designed for the web.
WebGL is base on OpenGL ES 2.0 and offers similar functionality.
WebGL acts as a rendering context for the HTML5 canvas element~\cite{web:html5canvas}, which supports programmatic rendering in web pages using different rendering APIs.
An existing 2D rendering context also exists known as CanvasRenderingContext2D, which provides the ability to draw in 2D on web pages.

WebGL enables a new generation of web games which are 3D, rather than 2D as current web games are.
Tech demos for WebGL show advanced effects such as bump-mapping~\cite{web:webglbumpmapping} and real-time water effects~\cite{web:webglwater}.
WebGL offers the easy delivery of applications, with no need for the user explicitly download their application to use it.

WebGL provides a set of flexible primitives which should be applicable in any use case.
The idea is that APIs will be developed on top of WebGL to provide support for specific areas.
Indeed a list of frameworks built on top of WebGL is available on the Khronos wiki~\cite{web:webglframeworks}.

However WebGL has problems which are inherent in its design.
Since it is web based, assets need to be transported over the network.
For large assets, this could take a long time.
With this in mind, we consider procedural content generation and how this might be used to alleviate the problem.

\subsection{WebGL Frameworks}
There are a variety of web frameworks to choose from nowadays.
This section will discuss a few of them and their merits for use with WebGL.

\paragraph{Javascript}
Javascript is the main language used by web browsers and is built into all modern browsers.
Javascript includes dynamic typing, objects, run-time evaluation.
Functions are first-class objects in javascript, and it is possible to have nested functions.
It uses prototype-based inheritance however, which most programmers would not be familiar with.

For use with webgl, the advantage is that javascript is the default language for interacting with webgl.
There are also numerous frameworks written in javascript to ease the use of webgl~\cite{web:threejs}\cite{web:copperlicht}. 

There are many quirks in javascript which some programmers abuse and according to Crockford~\cite{web:javascriptbadparts}
It's lack of strong typing and the leniancy of some web browsers on some javascript errors can make it difficult to troublshoot issues.

Javascript was not chosen because we needed a language that would suit for both desktop and web applications, which javascript is not suited for.

\paragraph{Coffee-script}
Coffee-script~\cite{web:coffeescript} is a language which compiles directly to javascript.
It includes classical inheritance as well as comprehensions and other features which improve upon javascript.
The output javascript passes jslint~\cite{web:jslint}, a javascript code-quality tool.

It also allows the user to use webgl natively and since it compiles to javascript, all of the existing javascript frameworks can easily be used.

However the syntax is very specific to javascript and the code cannot be reused outside of the web, which was essential for parts of thie project.

\paragraph{Processing.js}
Processing.js~\cite{web:processingjs} is a framework for executing programs written in the Processing~\cite{web:processing} language in the web browser.
Processing is a java-like language for executing small programs known as ``sketches''.
The advantage to processing is that it enables quick demos to be programmed, as a lot of the details of drawing are abstracted away from the programmer.
It has support for WebGL however it does not give the programmer control over the lower-level details of optimisation which may be needed to get maximum performance from this project.

\paragraph{Gwt}
Gwt~\cite{web:gwt} is a framework for creating web applications in the Java programming language~\cite{web:java}.
The Java code is compiled to javascript as with Coffee-script.
The advantage to using java is that it is statically typed, is well-known and there are many existing tools to use with it.
It also allows for modern compilation tools such as Apache Maven~\cite{web:maven} to be used because of the maven gwt plugin~\cite{web:mvngwtplugin}.
This allows the codebase to be used with many different IDEs and development environments.
It also allows for the easy deployment of the code to web servers using Apache Tomcat~\cite{web:tomcat}, which is very straightforward as maven can deploy straight to tomcat using the ``tomcat:deploy'' target~\cite{web:mvntomcatplugin}.

It is difficult to use javascript libraries with Gwt, requiring the writing of a jsni wrapper~\cite{web:jsni} to communicate back and forth.

There are modules for use with webgl with Gwt.
The most mature of which is gwtgl~\cite{web:gwtgl}.

GwtGL was chosen for this project because of the use of java principally which the author was already familiar with and which is also used with processing, which was used for the design aspects of this project.

\section{WebGL Raw Performance}
At the beginning of the project, it was not known whether WebGL was a bottleneck or not.
I wanted to figure out how many triangles WebGL could handle before the framerate dropped to levels which were too low.
The test I divised was to present a single quad on the screen, subdivided into smaller quads, based on the subdivision level.
This concept is illustrated in Figure~\ref{fig:webgl_perf}.

This test involved deploying the code locally on a machine with the following specifications:
\begin{itemize}
	\item Gwt version: Gwt 2.3.0
	\item GwtGL version: GwtGL 0.9-SNAPSHOT
	\item Web Browser: Chromium 15.0.849.0 (Developer Build 0 Linux)
	\item Web Server: Apache Tomcat/5.5.33
	\item OS: Linux 3.0-ARCH x86\_64
	\item GPU: Palit Nvidia GeForce GTS 250 1024MB GDDR3 PCI-Express Graphics Card
	\item CPU: Intel Core i7 920 D0 Stepping (SLBEJ) 2.66Ghz (Nehalem)
	\item Mem: Corsair XMS3 4GB (2x2GB) DDR3 PC3-10666C9 1333MHz Triple Channel
	\item HDD: Seagate Barracuda 7200.12 500GB SATA-II 16MB Cache
\end{itemize}

The graph of how the framerate is affected by the number of polygons on the screen is shown in Figure~\ref{fig:webgl_perf_graph}.
The numbers shown are the averages of hundreds of samples for each scene.
As can be seen even with 6000000 polygons on the screen at once, the framerate stays about 40 frames per second.
I was unable to test any more due to the polygon size being too great to be loaded in by the javascript in a small enough time.
However this proved to me that the bottleneck would not be WebGL itself, but the javascript code which loads assets.

\begin{figure}
  \centering
  \subcaptionbox{Level 1 subdivision}{\includegraphics[width=0.5\textwidth]{images/webgl_perf0}}
  \subcaptionbox{Level 2 subdivision}{\includegraphics[width=0.5\textwidth]{images/webgl_perf1}} 
  \caption{Illustrating how the subdivision of the quad performance test works}
  \label{fig:webgl_perf}
\end{figure}

\begin{figure}
  \centering
  \includegraphics[width=0.6\textwidth]{images/webgl_perf_graph}
  \caption{WebGL Framerate as the number of polygons on the screen at once increases}
  \label{fig:webgl_perf_graph}
\end{figure}


\section{Procedural Content Generation (PCG)}
\label{sec:pcg}
The generation of procedural content is important for many real-time applications. 
As we will see in this section it has been used to generate vast cities in real-time, the interiors of buildings and graphics effects on standard hardware.
Procedural Content Generation has the potential to 

\subsection{City Generation}
Recently, there has been much interest in the procedural generation of cities.
Ma\"{i}m et al.~\cite{maim2007populating} demonstrated how it is possible to recreate the population of historic cities using procedural techniques.
The use of procedural techniques to generate the crowds in Pompeii allowed realtime simulation with great variety in character representation.
This would not have been possible with traditional techniques.

The Metropolis project~\cite{web:metropolis} investigates the simulation of crowds in a modern city context.
Members of the crowd are modelled based on a variation of some template.
Different clothes are applied to the same templates to give the illusion of variation in the character models~\cite{mcdonnell2007pipeline}.
People are represented as agents which react to their environment~\cite{ulicny2002towards}.
In this way crowds are simulated using a variety of procedural techniques.

\begin{figure}
  \centering
    \includegraphics[width=0.7\textwidth]{images/metropolis}
  \caption{Screenshot of the Metropolis program in action}
\end{figure}

CityEngine~\cite{parish2001procedural} is an example of how procedural techniques can be applied to generates environments of great depth.
Using a combination of extended L-systems and self-sensitive L-systems, roads are generated which realistic simulate that of a target city.
Plans of buildings are generated between road segments in a recursive subdivision scheme, which discards inaccessible buildings.
The models of buildings are generated using an L-system using the bounding box of the building as the axiom of the L-system.
A technique known as \emph{layered grids} is proposed for the procedural generation of interesting textures for buildings.

\begin{figure}
  \centering
    \includegraphics[width=0.7\textwidth]{images/cityengine}
  \caption{Above: Street plan generated by CityEngine. Below: Actual plan of central manhattan.}
\end{figure}


Wonka et. al present a method for automatically modelling architecture~\cite{wonka2003instant} which generates a wide range of architectural features.
A new type of design grammar known as a ``Split grammars'' is used to derive building designs.
This allows the restriction of types of allowed rules. 
These restrictions make the grammars powerful enough for the modelling of buildings.
A parameter matching system allows the user to specify multiple high-level design goals so that the output appears consistent.
Control grammars are introduced which are simple context free grammars which handle spatial ideas in an orderly way which corresponds to architectural principles.
This paper's method gives a useful insight into how architectural features may be used to further enhance the generation of buildings in a procedural way.


\subsection{Building Interiors}
CityEngine provides a method for generating the exteriors of buildings, however this project will focus on the generation of indoor environments using procedural techniques.
Greuter et al. provide methods for generating\cite{greuter2003real} floor plans of buildings in a procedural fashion.
Plans are generated by merging various polygonal shapes in a pseudo-random fashion to generate a final plan of a series of floors.
These floor plans are extruded into the 3rd axes and the outdoor buildings models are created from this.
The floors are not populated with rooms however as the indoor environments are not meant to be seen.

So et al. use wall extrusion for generating 3D indoor environments from floor plans~\cite{so1998reconstruction}.
These generated indoor environments contain rooms.
So et al. apply the technique to a CAD tool, but the technique could easily be applied to an OpenGL-based environment also.

Hahn et al present a novel approach to the generation of virtual building interiors in real-time~\cite{hahn2006persistent}.
The approach uses a lazy generation scheme which is advantageous as it means that the amount of memory used is small. 
They divide the interiors of buildings into temporary regions and built regions.
Built regions contain the final visible product of the generator and hold the geometry needed for collision detection and rendering.
Temporary regions are placeholder regions which are turned into built regions in a lazy fashion.
Temporary regions are populated when the player enters them, via a portal system.
The generation of built regions is split into stages to simplify the implementation:

\begin{enumerate}
	\item Building Setup : Anything which effects multiple floors is generated at this stage. Elevators and stairs are included as well as global textures.
	\item Floor Division : Divides the building into evenly spaced floors. Floors are then divided into 2 parts.
	\item Hallway Division : Hallways are constructed by dividing the regions around other blocks which can be rectangular loops or straight segments of hallway. 
	\item Room Cluster Division : Regions between hallways are divided into rooms. 
	\item Built Region Generation : The geometry of the room is created and it is populated with objects. This is only done if the room contains a portal.
\end{enumerate}

The oldest generated built regions are periodically deleted to control memory. 
Newly generated built regions are put into a LRU cache so that they can be easily recalled if the player moves back into the region.

\subsection{Demoscene}
The demoscene is a community of computer programmers who specialise in making impressive visual and audio effects.
There is usually an emphasis on small code size such as the 4K competitions which has the restriction on a code size of 4KB~\cite{web:demoscene4k}.
These restricted sized demos are usually referred to as Intros.

Many demoscene programmers have applied their expertise to game-related applications.
One notable example is .theprodukkt~\cite{web:theprodukkt}, who developed the .kkrieger application~\cite{web:kkrieger}.
.kkrieger displays graphics of a level comparable to Doom3 in an executable which is 96K in size.
A comparison of .kkrieger with Doom 3 can be seen in \ref{fig:kkriegerdoomcomp}.
It achieves the results seen using a variety of procedural techniques, many of which will be applicable for this project.
An example is the procedural generation of textures.
It is unfortunant however that Demoscene programmers rarely release the code that they use to produce the effects shown.
This is the case with .kkrieger also.

\begin{figure}
  \centering
  \subcaptionbox{.kkrieger}{\includegraphics[width=0.5\textwidth]{images/kkrieger}}                
  \subcaptionbox{Doom 3}{\includegraphics[width=0.5\textwidth]{images/doom3}}                
  \caption{Comparison screenshots of Doom 3 and kkrieger}
  \label{fig:kkriegerdoomcomp}
\end{figure}


  \chapter{Design}
 - Main design goals:   
    >> Fast loading times
    >> Nice-looking scenes
    >> Scenes which might resemble a game
    >> Ability to assess performance relative to a non-procedural version
    >> Easy to edit design program
 - How were design goals addressed?
    >> Seperate of procedural and non-procedural versions.
    >> Pipeline for generation of content.
    >> Created texture generators from scratch
    >> Perlin Nose done in frag shader (V.low latency)
    >> Design program written in processing which makes the results quick to see and is also easy to edit

 - What the influences were ?
    >> Architecture
        -> Christopher Alexander's "A Pattern Language"
        -> Plan generation relies on this heavily
    >> CityEngine
    >> Quake
        -> GL Quake
        -> Quake under gwt

  \chapter{Implementation}

 - Details of the the codebase.
    >> UML
    >> Seperation of procedural and non-procedural
    >> Processing application vs GwtGL demo

\section{Geometry Generation}

 - Plan generation
 - Extrusion

 - Problems:
    >> Time constraints
        -> No time to implement all of the plan generation
        -> No progressive meshes

\section{Texture Generation}
 - Perlin Noise
 - Bump mapping
 - Lighting

 - Problems:
    >> WebGL current limitations.
        -> No noise() function
        -> OpenGL ES 2.0 extensions are generally not supported yet.

  \chapter{Evaluation}

 - Performance test results.
    >> Procedural vs non-procedural
	>> Different connections
	>> Different Loading times
 - How easy it is to create a variety of scenes.
 - Ease of use.

  \chapter{Conclusions And Further Work}
 - How do scenes resemble a real environment?
 - How do scenes resemble a game?

\section{Suitability of WebGL as platform for Games}

\section{Further Work}

\subsection{Progressive Meshes}

\subsection{More Advanced Plan Generation}

\subsection{Artistic Authoring}


\addcontentsline {toc}{chapter}{Appendices}       %% Force Appendices to appear in contents
\begin{appendix}
 \chapter*{Appendix}
 - Details of the the codebase.
    >> UML


...

% \include{appendix2}
\end{appendix}


%\addcontentsline {toc}{chapter}{Bibliography}     %% Force Bibliography to appear in contents

%\begin{thebibliography}{ieeetr}                   %% Start your bibliography here; you can
\bibliographystyle{ieeetr}                               %% also use the \bibliography command
\bibliography{thesis}                               %% also use the \bibliography command
%\end{thebibliography}                             %% to generate your bibliography.


\end{document}                                    %% END THE DOCUMENT
